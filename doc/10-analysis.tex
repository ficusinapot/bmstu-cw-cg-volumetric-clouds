\chapter{Аналитическая часть}
В аналитической части будут формализованы задачи и объекты сцены, определены геометрические и оптические характеристики объектов сцены. Также будут проанализированы и описаны алгоритмы, используемые для визуализации ландшафтной сцены с облаками. Будут установлены допустимые диапазоны и ограничения, накладываемые на входные данные.

\section{Формализация задачи и объектов}

Объектами сцены являются:
\begin{enumerate}
	\item \textbf{Облака (облачный пейзаж)}
	\begin{itemize}
		\item Высота, на которой находятся облака;
		\item Скорость движения облаков по горизонту;
		\item Кучность: степень сжатия и плотности облаков, что влияет на их внешний вид и отбрасываемую тень.
		\item Плотность: определяет, сколько солнечного света облака могут заблокировать, что влияет на освещение ландшафта.
	\end{itemize}
	\item \textbf{Ландшафт (ландшафтный пейзаж)} -- 
	\begin{itemize}
		\item Рельеф: плоский равнинный.
		\item Материалы и текстуры: характеристики поверхности, такие как цвет и отражательная способность.
		\item Освещение от солнца и теней: ландшафт получает освещение, которое зависит от плотности облаков и положения солнца, а также отбрасываемых теней.
	\end{itemize}
	\item \textbf{Бесконечно удаленный источник света (солнце)} --
	\begin{itemize}
		\item Расположение: определяется положением на небесной сфере. Положение солнца влияет на длину и направление теней.
		\item Интенсивность: определяет, насколько ярко освещен ландшафт, также зависит от плотности облаков.
	\end{itemize}
	\item \textbf{Наблюдатель (камера)} --
	\begin{itemize}
		\item Расположение: координаты и угол обзора камеры, позволяющие наблюдать сцену с разных ракурсов.
		\item Поле зрения: угол обзора, влияющий на широту сцены.
	\end{itemize}
\end{enumerate}

Определение диапазонов и ограничений:
\begin{itemize}
	\item \textbf{Положение солнца:} угол наклона от $0^{\circ}$ до $90^{\circ}$ над горизонтом и азимутальный угол от $0^{\circ}$ до $180^{\circ}$.
	\item \textbf{Пространственное перемещение:} только для таких объектов, как камера и солнце.
\end{itemize}

\section{Алгоритмы генерации облаков}
Существует несколько подходов к реализации облаков \cite{unigine_volumetric_clouds_2022,guerrilla_volumetric_cloudscapes_2023,  sym10040125}:	\begin{itemize}
\item \textbf{Геометрический:} облака представляют собой, например, набор треугольников, сфер или прямоугольников. Геометрический подход к созданию облаков имеет смысл в определенной стилистике изображения \cite{unigine_volumetric_clouds_2022}.
\item \textbf{Двумерная текстура:} простой и малозатратный подход, но такая статичная картинка имеет смысл только как дальнеплановые статичные изображения, через которые, например, нельзя пролететь сквозь. К тому же такие облака не могут производить тени \cite{unigine_volumetric_clouds_2022}.
\item \textbf{Объемные} \textit{(volumetric)}: динамические облака, с которыми можно взаимодействовать и которые способны производить тени \cite{shadows2023, guerrilla_volumetric_cloudscapes_2023}. Именно поэтому такие облака будут реализованы в данной работе.
\end{itemize}

Исходя из требований к алгоритму, выдвигаемых в современной игровой индустрии \cite{unigine_volumetric_clouds_2022,guerrilla_volumetric_cloudscapes_2023,  sym10040125}, условия, которые будут использованы в данной работе:
\begin{itemize}
	\item Облака должны быть объемные;
	\item Облака должны генерироваться процедурно;
	\item Должен быть быстродействующим.
\end{itemize}
\subsection{Жидкостная симуляция} 

Использование жидкостной симуляции для создания объемных облаков: создать простые объекты (сферы, шары), вокселизировать их и рассматривать их как жидкость, получая похожие на объемные облака фигуру \cite{guerrilla_volumetric_cloudscapes_2023}.
Современные физические модели облаков, основаны на решении \textit{уравнений Навье-Стокса}~\cite{sym10040125}, что влечет за самой следующие недостатки~\cite{sym10040125}:\\
\begin{itemize}
	\item Алгоритм медленный;
	\item Сложность контроля генерации;
	\item Сложность реализации.
	\item Не используется на практике~\cite{unigine_volumetric_clouds_2022}
\end{itemize}
\subsection{Воксельная генерация}
Алгоритм заключается в генерации ограничивающего параллелепипеда (bounding box), состоящего из вокселей, хранящих информацию о цвете~\cite{guerrilla_volumetric_cloudscapes_2023}.
Преимущества:
\begin{itemize} 
	\item Хорошо сочетается с алгоритмом построением теней
\end{itemize}
Недостатки: 
\begin{itemize} 
	\item Высокие затраты памяти; 
	\item Сложность обработки большого количества вокселей в реальном времени; 
	\item Необходимость оптимизаций для обработки больших объемов. 
	\item Не используется на практике~\cite{guerrilla_volumetric_cloudscapes_2023}
\end{itemize}
\subsection{Генерация на основе обратной трассировки лучей}
Из точки наблюдателя для каждого пикселя грани высчитывается его итоговый цвет~\cite{Patapom2013, sym10040125}.
Алгоритм также опирается на ограничивающий параллелепипед, но вместо этого визуализируются лишь видимые грани параллелепипеда. Вместо вычисления каждого вокселя, алгоритм ориентируется на пиксели, видимые пользователю, и рассчитывает итоговые цвета только для них. 
\begin{itemize} 
	\item Хорошо сочетается с алгоритмом построением теней;
	\item Меньшие затраты памяти; 
	\item Сниженные вычислительные затраты благодаря обработке только видимых пикселей.
	\item Рекомендованы к использованию на практике~\cite{guerrilla_volumetric_cloudscapes_2023, sym10040125, windahl_real_time_2018}
\end{itemize}
Генерация на основе обратной трассировки лучей показывает преимущество перед воксельной и жидкостной генерациях, так как обрабатывает только видимые пиксели, что снижает вычислительные затраты и экономит память, что необходимо при формировании динамического изображения.

\section{Модели освещения}
\subsection{Закон Бугера~---~Ламберта~---~Бера}
Для облаков некоторая часть света рассеивается от направления распространения, а еще большее количество поглощается каплями воды и молекулами озона, но остается часть, которая продолжает движение без изменений~\cite{guerrilla_volumetric_cloudscapes_2023, sym10040125}.

\textit{Закон Бугера---Ламберта---Бера} определяет ослабление пучка света при поглощении средой.
\begin{equation}
	\label{eq:beers-law}
	I_l=I_{o}e^{-k_{\lambda }l},
\end{equation} где 
{$I_{0}$} — интенсивность света на входе в вещество, 
$k_\lambda$ — показатель поглощения.
\subsection{Фазовая функция Хеньи — Гринстейна}
Облака представляют собой анизотропную среду (среда, где физические свойства: показатели преломления, скорость распространения и пр. -- различаются в различных направлениях внутри этой среды) из-за того, что облака представляют собой капли жидкой воды и кристаллов ледяного льда. Для описания этого используют фазовую функцию (индикатриса) Хеньи — Гринстейна~\cite{guerrilla_volumetric_cloudscapes_2023, sym10040125}

\textit{Фазовая функция Хеньи — Гринстейна} определяет угловое распределение интенсивности:
\begin{equation}
	\label{eq:henyey-greenstein}
	P(g, \theta) = \frac{1 - g^2}{(1 + g^2 - 2g \cos\theta)^{3/2}}
\end{equation} где 
$\theta$ — угол рассеяния, который определяется как угол между направлением распространения исходного и рассеянного света и $g$ — параметр асимметрии, который описывает среднее значение косинуса угла рассеяния.

\section*{Вывод}
В аналитической части формализованы задачи и объекты сцены, определены геометрические и оптические характеристики объектов сцены. Также проанализированы и описаны алгоритмы, используемые для визуализации ландшафтной сцены с облаками. Установлены допустимые диапазоны и ограничения, накладываемые на входные данные.
Был выбран алгоритм использующий обратную трассировку лучей для генерации объемных облаков.%, а также алгоритм для построения ландшафта с помощью аппроксимацией примитивами.
