\ssr{ВВЕДЕНИЕ}
\textbf{Компьютерная графика} -- совокупность методов и средств преобразования в графическую форму и из графической формы с помощью ЭВМ \cite{kurov2024lections}. Конечным продуктом компьютерной графики является изображение \cite{rodgers1989algorithms}. Ключевые моменты, которые компьютерная графика рассматривает -- как \cite{rodgers1989algorithms}
\begin{itemize}
	\item изображения представляются в компьютерной графике;
	\item изображения готовятся для визуализации;
	\item предварительно подготовленные изображения рисуются;
	\item осуществляется взаимодействие с изображением.
\end{itemize}
\textbf{Цель работы} -- разработка программного обеспечения для визуалации динамической ландшафтной сцены с облаками.\\
для достижения поставленной цели требуется решить следующие задачи:
\begin{itemize}
	\item изучить предметную область;
	\item спроектировать программное обеспечение;
	\item выбрать средства реализации программного обеспечения и создать его;
	\item провести исследование разработанного программного обеспечения.
\end{itemize}

